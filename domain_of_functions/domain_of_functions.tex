\section{Domain of Functions}

This section deals mainly with polynomial, rational, and even parity radical functions, as they provide the example best for obtaining an intuition for the concept of domain.

\begin{example}
We have an example of the identity function, in the attenuated domain:
\begin{align*}
    f(x) = x \hspace{20pt} x \in [-2, 2]
\end{align*}
in which the blue line represents the function, and the red line represents the $x$ values on which $f$ is acting. The collection of these $x$ values is known as the domain of $f$. Think about it as if the function $f$ is pulling $x$ from the red line to the vertical position at which $f$ wishes to settle. 

\vspace{0.5in}
\resizebox{30em}{30em}{%
\begin{tikzpicture}[scale=\textwidth/5.2cm]
    % title and axes
    \node at (-1, 1.1) {$f(x) = x \hspace{20pt} x \in [-2, 2]$};
    \draw (-2, 0) -- (2, 0)
        node[right] {$x$};
    \draw (0, -2) -- (0, 2)
        node[above] {$f(x)$};
    % graph
    \draw[blue, very thick] plot[smooth] file {domain_of_functions/identity_neg2_2.table};
    \draw[red, very thick] plot[smooth] file {domain_of_functions/domain_identity_neg2_2.table};
    \draw[dotted, ->] (2, 0) -- (2, 1.95);
    \draw[dotted, ->] (1.5, 0) -- (1.5, 1.45);
    \draw[dotted, ->] (1, 0) -- (1, 0.95);
    \draw[dotted, ->] (0.5, 0) -- (0.5, 0.45);
    \draw[dotted, ->] (-0.5, 0) -- (-0.5, -0.45);
    \draw[dotted, ->] (-1, 0) -- (-1, -0.95);
    \draw[dotted, ->] (-1.5, 0) -- (-1.5, -1.45);
    \draw[dotted, ->] (-2, 0) -- (-2, -1.95);
\end{tikzpicture}
}
\end{example}

\begin{example}
We look at the same function, but instead of the entirety of our attenuated domain, we have pieced the attenuated domain into two sub-intervals:
\begin{align*}
    f(x) = x \hspace{20pt} x \in [-2, 0] \cup \Big[\dfrac{1}{2}, 2\Big]
\end{align*}
in which the blue line represents the function, and the red line represents the $x$ values on which $f$ is acting. Because there are not $x$ values in both the domain of $f$ (Dom($f$)) and the interval $(0, 1/2)$, there is nothing there for $f$ to act on. Thus, the function $f$ (the blue line) contains no information along that interval $(0, 1/2)$.

\vspace{0.5in}
\resizebox{30em}{30em}{%
\begin{tikzpicture}[scale=\textwidth/5.2cm]
    % title and axes
    \node at (-1, 1.1) {$f(x) = x \hspace{20pt} x \in [-2, 0] \cup \Big[\dfrac{1}{2}, 2\Big]$};
    \draw (-2, 0) -- (2, 0)
        node[right] {$x$};
    \draw (0, -2) -- (0, 2)
        node[above] {$f(x)$};
    % graph
    \draw[blue, very thick] plot[smooth] file {domain_of_functions/identity_neg2_2_discontinuous_piece_0.table};
    \draw[blue, very thick] plot[smooth] file {domain_of_functions/identity_neg2_2_discontinuous_piece_1.table};
    \draw[red, very thick] plot[smooth] file {domain_of_functions/domain_identity_neg2_2_discontinuous_piece_0.table};
    \draw[red, very thick] plot[smooth] file {domain_of_functions/domain_identity_neg2_2_discontinuous_piece_1.table};
    \draw[dotted, ->] (2, 0) -- (2, 1.95);
    \draw[dotted, ->] (1.5, 0) -- (1.5, 1.45);
    \draw[dotted, ->] (1, 0) -- (1, 0.95);
    \draw[dotted, ->] (0.5, 0) -- (0.5, 0.45);
    \draw[dotted, ->] (-0.5, 0) -- (-0.5, -0.45);
    \draw[dotted, ->] (-1, 0) -- (-1, -0.95);
    \draw[dotted, ->] (-1.5, 0) -- (-1.5, -1.45);
    \draw[dotted, ->] (-2, 0) -- (-2, -1.95);
\end{tikzpicture}
}
\end{example}

\begin{recall}
If the function $f$ is
\begin{itemize}
    \item rational, then any $x$ value that makes the denominator of $f$ equal to $0$ will not belong to Dom($f$).\\
    \item even parity radical (think square root, fourth root, etc.), then any $x$ value that makes the radicand less than $0$ will not belong to Dom($f$).
\end{itemize}
\end{recall}